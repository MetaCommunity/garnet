\make{manual}
\disable{figurecontents}
\libraryfile{Garnet}
\string{TitleString = `Garnet Toolkit'}
\use{Bibliography = `garnet.bib'}

\begin{titlepage}
\begin{titlebox}
% \comment{This is to make sure it does not affect the previous page,
% 	e.g. putting it at the top of the file won't cut it:}
\vspace{0.6 inch}
\style{PageNumber {\tt\char`\<}\1>}
\bg{Hints on Making Garnet Programs Run Faster}

{\bf Brad A. Myers}
\vspace{0.3 line}
\value{date}
\end{titlebox}
\vspace{0.5 inch}
{\bf Abstract}

\begin{text}
This chapter discusses some hints about how to make Garnet programs
run faster.  Most of these techniques should not be used until your
programs are fully debugged and ready to be shipped.
\vspace{0.7 inch}
\begin{center}
Copyright \j{w} 1990 - Carnegie Mellon University\end{center}

This research was sponsored by the Defense Advanced Research Projects
Agency (DOD), ARPA Order No. 4976, Amendment 20, under contract
F33615-87-C-1499,
monitored by the Avionics Laboratory, Air Force Wright Aeronautical
Laboratories, Aeronautical Systems Division (AFSC), Wright-Patterson AFB,
Ohio 45433-6543.

The views and conclusions contained in this document are
those of the authors and should not be interpreted as representing the
official policies, either expressed or implied, of the Defense Advanced
Research Projects Agency or the US Government.

\end{text}
\end{titlepage}


\string{overview = `1'} % \comment{26 pages}
\string{overview-first-page = `3'}
\string{apps = `27'} % \comment{12 pages}
\string{apps-first-page = `29'}
\string{tour = `41'} % \comment{20 pages}
\string{tour-first-page = `43'}
\string{tutorial = `61'} % \comment{42 pages}
\string{tutorial-first-page = `63'}
\string{kr = `101'} % \comment{52 pages}
\string{kr-first-page = `103'}
\string{opal = `151'} % \comment{70 pages}
\string{opal-first-page = `153'}
\string{inter = `219'} % \comment{78 pages}
\string{inter-first-page = `221'}
\string{aggregadgets = `295'} % \comment{54 pages}
\string{aggregadgets-first-page = `297'}
\string{gadgets = `347'} % \comment{116 pages}
\string{gadgets-first-page = `349'}
\string{debug = `461'} % \comment{20 pages}
\string{debug-first-page = `463'}
\string{demos = `481'} % \comment{10 pages}
\string{demos-first-page = `483'}
\string{sampleprog = `491'} % \comment{14 pages}
\string{sampleprog-first-page = `493'}
\string{gilt = `505'} % \comment{20 pages}
\string{gilt-first-page = `507'}
\string{c32 = `525'} % \comment{12 pages}
\string{c32-first-page = `527'}
\string{lapidary = `537'} % \comment{36 pages}
\string{lapidary-first-page = `539'}
\string{hints = `573'} % \comment{8 pages}
\string{hints-first-page = `575'}
\string{GlobalIndex = `580'}

\set{page=hints-first-page}
\artsection{Introduction}

An important goal of Garnet has been to create a system that is as
efficient as possible.  For example, users should notice that version 2.2
is about two or three times faster than 2.1.  Now that
people are writing large-scale systems using Garnet, a number of
things have been learned about how to make Garnet programs run faster.
This chapter collects a number of hints about how to write efficient
Garnet code.  If you have ideas about how to make the underlying
Garnet system run faster, or new hints to add to this section, please
let us know.

The ideas in this chapter are aimed at producing the final production
version of your system.  Therefore, we feel that you should not worry
about the comments here during early development.  For example,
turning off the debugging and testing information is likely to make
your development more difficult.  Also, declaring constants makes
changing code more difficult.  Generally, you should get your system
to a fairly well-debugged state before applying these ideas.

Of course, the easiest way to make Garnet run faster is to get a
faster machine and/or more physical memory.  With SPARC IIs and HP
Snakes becoming more prevalent, and 100 mip machines like the DEC
Alpha around the corner, we see expect that the next generation of
applications will have much less of a problem with achieving adequate
performance.

\artsection{General}
Ideas in this section are relevant to any code written in Lisp, not
just Garnet code.  Some of these may seem obvious, but we have seen
code that violates many of them.

\begin{itemize}
\item Be sure to compile all your files.

\item 
The variable \pr{user::*default-garnet-proclaim*}, which is defined in
\pr{garnet-loader.lisp}, provides some default compiler optimization values
for Allegro, Lucid, CMU, LispWorks, and MCL lisp implementations.  The default
gives you fast compiled code with verbose debugging help.  You can \pr{setf}
this variable before loading (and compiling) Garnet to override the default
proclamations, if you want to sacrifice debugging help for speed:
\begin{programexample}
(PROCLAIM '(OPTIMIZE (SPEED 3) (SAFETY 0) (SPACE 0)
	             (COMPILATION-SPEED 0)))
\end{programexample}


\item Fundamental changes in underlying algorithms will often overcome any
local tweaking of code.  For example, changing an algorithm that
searches all the objects to one that has a pointer or a hash table to
the specific object can make an application practical for large
numbers of objects.

\item Use a fast Lisp system.  We have found that Allegro Version 4.2 is
much faster than Allegro V3.x.  Also, Allegro and Lucid are much
faster than KCL and AKCL on Unix machines.

\item Most systems have specialized commands and features for making smaller
and faster systems.  For example, if you are using
Allegro, check out PRESTO, which tries to make the run-time image
smaller.  One user reported that the `reorganizer' supplied with
Lucid, the CPU time used decreased about 10-20\%, and the overall
time for execution dropped by about 30\%.
We have found that the tracing tools supplied by vendors to
find where code is spending its time are mostly worthless, however.

\item Beware of Lisp code which causes CONS'ing.  Quite often, the most
natural way to write Lisp code is the one that creates a lot of
intermediate storage.  Unfortunately, this may result in severe
performance problems, as allocating and garbage-collecting storage is
among the slowest operations in Lisp.  The recommendations below apply
to all of your code in general, but in particular to code that
may be executed often (such as the code in certain formulas which need
to be recomputed many times).

\item As a rule, mapping operations (like \pr{mapcar}) generate garbage in most Lisp
implementations, because they create temporary (or permanent) lists of
results.  Most mapping operations can be rewritten easily in terms of
DO, DOLIST, or DOTIMES.

\item Handling large numbers of objects with lists is generally expensive.
If you have lists of more than a few tens of objects, you should
consider using arrays instead.  Arrays are just as convenient as lists,
and they require much less storage.  If your application needs
variable numbers of objects, consider using variable-length arrays
(possibly with fill pointers).

\item Declare the types of your variables and functions (using DECLARE and
PROCLAIM).

\item Some Lisp applications will give you warnings or notes about Lisp
constructs that are potentially inefficient.  In CMU Common Lisp, for
example, setting SPEED to 3 and COMPILATION-SPEED to 0 generates a
number of messages about potentially inefficient constructs.  Many
such inefficiencies can be eliminated easily, for example by adding
declarations to your code.

\item Wrap all lambdas in \#' rather than just ' (in CLtL2 the \# is no longer
optional).  This
comes up in Garnet a lot in final-functions for interactors and
selection-functions for gadgets.  Note, in the \pr{:parts} or
\pr{:interactors} parts of aggregadgets or aggrelists, use \pr{,\#'}
(comma-number-quote) before lambdas and functions.

\item You can save an enormous amount of time loading software if you make images
of lisp with the software already loaded.  For example, if you start lisp and
load Garnet, you can save an image of lisp that can be restarted later with
Garnet already loaded.  We have simplified this procedure by providing the
function \pr{opal:make-image}.  If you want to make images by hand, you will
have to use \pr{opal:disconnect-@{\tt\char`\|}garnet} and \pr{opal:reconnect-@{\tt\char`\|}garnet}
to sever and restore lisp's connection with the X server.  All of these
functions are documented in the Opal manual.

\item It may help to reboot your workstation every now and then.  This will reset
the swap file so that large applications (like Garnet) run faster.
\end{itemize}



\artsection{Making your Garnet Code Faster}
This section contains hints specifically about how to make Garnet code
faster.

\begin{itemize}
\item The global switch \pr{:garnet-debug} can be removed from the
\pr{*features*} list to cause all the debugging and demo code in Garnet to
be ignored during compiling and loading.  This will make Garnet slightly
smaller and faster.  The \pr{:garnet-debug} keyword is pushed onto the
\pr{*features*} list by default in \pr{garnet-@{\tt\char`\|}loader.lisp}, but you can
prevent this by setting \pr{user::Garnet-Garnet-Debug} to NIL before compiling
and loading Garnet.  Garnet will need to be recompiled with the new
\pr{*features*} list, so that the extra code will not even get into the
compiled binaries.  Of course, you will lose functions like
\pr{inter:trace-inter}.

\item Turn off KR's type-checking by setting the variable \pr{kr::*types-enabled*}
to NIL.  Note: the speed difference may be imperceptible, since the type
system has been implemented very efficiently (operations are only about 2\%
slower with type-checking).

\item If you have many objects in a window, and an interactor only works on a
small set of those objects, then the small set of objects should be in
their own aggregate or subwindow.  This will cause Opal's \pr{point-in-gob}
methods run faster, which identify the object that you clicked on.  When
objects are arranged in an orderly aggregate hierarchy, then the
\pr{point-in-gob} methods can reject entire groups of objects, without checking
each one separately, by checking whether a point is inside their
{\it aggregate}'s bounding box.  For example, in \pr{demo-motif} the scroll bars
are in their own aggregate.  Putting objects in a seperate subwindow is even
faster, since the coordinates of the click will only be checked against objects
in the same window as the click.

\item Use \pr{o-formula}s instead of \pr{formula}s.  O-formulas are compiled
along with the rest of the file, whereas formulas are compiled at load-
or run-time, which is much slower.

\item Try not to use formulas where not really needed.  For example, if the
positions of objects won't change, use expressions or numbers instead
of formulas to calculate them.

\item Try to eliminate as many interactors as possible.  Garnet must
linearly search through all interactors in each window.  To see how
many interactors are on your window, you can use
\pr{(inter:print-inter-levels)}.  If this is a long list, then try to
use one global interactor with a start-where that includes lots of
objects, rather than having each object have its own interactor.  This
can even work if you have a lot of scattered gadgets.  For example, if
you have a lot of buttons, you can use a button-panel and override the
default layout to individually place each button.

\item The \pr{fast-redraw} property of graphical objects can be set to make
objects move and draw faster.  This can be used in more cases than
with previous versions of Garnet, but it is still restricted.  See the
fast-redraw section of the Opal manual.

\item Aggrelists are quite general, and have a lot of flexibility.  If you
don't need this flexibility, for example, if your objects will always
be in a simple left-aligned column, it will be more efficient to place
the objects yourself, or create custom formulas.

\item If you are frequently destroying and creating new objects of the same
type, it is more efficient to just keep a list of objects
around, and re-using them.  Allocating memory in Lisp is fairly
expensive.

\item 
If you are deleting a number of objects at the same time, first set
the window's \pr{:aggregate} slot to NIL and update the window.
Then, when you are done destroying, set the aggregate back and update
again.  For example, to destroy 220 rectangles on a Sparc, removing the
aggregate reduced the time from 11.8 to 2.4 seconds (80\%)!
So your new code should be:
\begin{programexample}
;; {\it Code fragment to quickly destroy all the objects within an aggregate.}
(let ((temp-agg (kr:gv my-window :aggregate)))
  (when temp-agg
    ;; First, temporarily remove the aggregate:
    (kr:s-value my-window :aggregate NIL)
    (opal:update my-window)
    ;; Now do the actual destroying:
    (dolist (object (kr:get-values temp-agg :components))
      (opal:destroy object))
    ;; Finally, restore the aggregate:
    (kr:s-value my-window :aggregate temp-agg)
    (opal:update my-window)))
\end{programexample}


\item If you have objects in different parts of the same window changing at
the same time, it is often faster to call update explicitly after one
is changed and before the other.  (This is only true if neither of the
objects is a fast-redraw object.  Many of the built-in gadgets are
fast redraw objects for this reason, so this usually is not necessary
for built-in gadgets.)  The reason for this problem is that Garnet
will redraw everything in a bounding box which includes all the
changed objects.  If the changed objects are in different parts of a
window, then everything in between will be redrawn also.  Ways around
this problem include calling update explicitly after one of the
objects changes, making one of the objects be a fast redraw object if
possible, moving the objects closer together if possible (so there
aren't objects in between), or putting the objects in separate
subwindows if possible (subwindows are updated independently).

\item Conventional object-oriented programming relies heavily on message
sending.  In Garnet, however, this technique is often less efficient
than the preferred Garnet programming style, which relies on slots and
constraints.  Rather than writing methods to get values from certain
slots in an object, for example, consider accessing those slots directly and
having a formula compute their value.   The Garnet style is more
efficient, since it avoid the message-sending overhead.  Because
Garnet provides a powerful constraint mechanism, the functionality
that would normally be associated with a method can typically be
implemented in a formula.

\item 
If you use the same formula in multiple places, it is more efficient
to declare a formula prototype, and create instances of it.  For
example:
\begin{programexample}
(defparameter leftform (o-formula (+ 10 (first (gvl :box)))))
;;{\it for every object}
(create-instance NIL {\it {\tt\char`\<}whatever>}
		 ...
		 (:left (formula leftform)))
\end{programexample}


\item If many objects in your scene have their own feedback objects, maybe
you can replace these with one global feedback object instead.  The
button and menu interactors can take a \pr{:final-feedback-obj}
parameter and will duplicate the feedback object if necessary.

\item If you have a lot of objects that become invisible and stay invisible
for a reasonable period if time, it might be better to remove them
from their aggregate rather than just setting their \pr{:visible}
slot.  There are many linear searches in Garnet that process all
objects in an aggregate, and each time it must check to see if the
objects are invisible.

\item It is slightly more efficient when you are creating a window at
startup, if you add all the objects to the top level aggregate
{\it before} you add the aggregate to the window.

\item The use of double-buffering doesn't make your applications
run faster (they actually run a little slower), but it usually
{\it appears} faster due to the lack of flicker.  See the section in the
Opal manual on how to make a window be double-buffered.
\end{itemize}

\artsection{Making your Binaries Smaller}
This section discusses ways to make the run-time size of your
application smaller.  This is important because when your system gets
big, it can start to swap, which significantly degrades performance.
We have found that many applications would be fast enough if they all
fit into physical memory, whereas when they begin swapping virtual
memory, they are not fast enough.

\begin{itemize}
\item Don't load the PostScript module or debugging code unless you need to.
Change the values of the appropriate variables in \pr{garnet-loader},
or set the variables before loading Garnet.  The values will not be overridden,
since they are defined with \pr{defvar} in \pr{garnet-loader}.

\item 
Declare constants where possible.  This allows Garnet to throw away
formulas, which saves a lot of run-time space.  All the built-in objects and
gadgets provide a \pr{:maybe-constant} slot, which means that you can
use \pr{(:constant T)} to make all the slots constant.
The \pr{:maybe-constant} will contain all of the slots discussed in the
manual as parameters to the object or gadget.  Of course, the slots
that allow the widget to operate (e.g., the buttons to be pressed or
the scroll-bar-indicator to move) are not declared constant.
Remember that only slots that don't change can be declared constant.
Therefore, if your gadget changes position or items or active or font after
creation, then you should \pr{:except} the appropriate slots.  For
example:
\begin{programexample}
(create-instance NIL gg:motif-radio-button-panel
  ;; {\it only the :active slot will change}
  (:constant '(T :except :active))
  (:left 10)(:top 30)
  (:items '(`Start' `Pause' `Quit')))
\end{programexample}
Several functions are discussed in the Debugging Manual (starting on page
\value{debug}) that are very helpful in determining which slots should be
declared constant.  The KR Manual describes the fundamentals of
constant declarations in detail.


\item Don't load gadget files you don't need.  Most Garnet applications
(like the demos), load only the gadgets they need, if they haven't
been loaded already.  This approach means that lots of gadgets you
never use won't take up memory.

\item Consider using \pr{virtual-aggregates} if you have a lot of similar
objects in an interface, such as lines in a map or dots on a graph.
This will decrease storage requirements significantly.

\item The variable \pr{kr::store-lambdas} can be set to NIL to remove the
storage of the lambda expressions for compiled formulas.  This will
save some storage, but it prevents objects from being stored to files.

\end{itemize}

