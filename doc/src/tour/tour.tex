\device{postscript}
\make{manual}
\disable{figurecontents}
\libraryfile{Garnet}
\string{TitleString = `Garnet Tour'}
\use{Bibliography = `garnet.bib'}
\begin{titlepage}
\begin{titlebox}
\vspace{0.6 inch}
\bg{On-line tour through Garnet}

{\bf Brad A. Myers}
\vspace{0.3 line}
\value{date}
\end{titlebox}
\vspace{0.5 inch}
\begin{center}
{\bf Abstract}\end{center}
\begin{text}
This document provides an on-line tour through some of the features of the
Garnet toolkit.  It serves as an introduction to the toolkit and how to
program with it.  This document and tour do {\it not} assume that the reader
has read the reference manuals.  The tour only assumes that the reader is
familiar with CommonLisp and has loaded the Garnet software.

\vspace{0.5 inch}
\begin{center}
Copyright \j{w} 1990 - Carnegie Mellon University\end{center}

This research was sponsored by the Defense Advanced Research Projects
Agency (DOD), ARPA Order No. 4976, Amendment 20, under contract
F33615-87-C-1499,
monitored by the Avionics Laboratory, Air Force Wright Aeronautical
Laboratories, Aeronautical Systems Division (AFSC), Wright-Patterson AFB,
Ohio 45433-6543.

The views and conclusions contained in this document are
those of the authors and should not be interpreted as representing the
official policies, either expressed or implied, of the Defense Advanced
Research Projects Agency or the US Government.

\end{text}
\end{titlepage}


\define{typ=ux}

\string{overview = `1'} % \comment{26 pages}
\string{overview-first-page = `3'}
\string{apps = `27'} % \comment{12 pages}
\string{apps-first-page = `29'}
\string{tour = `41'} % \comment{20 pages}
\string{tour-first-page = `43'}
\string{tutorial = `61'} % \comment{42 pages}
\string{tutorial-first-page = `63'}
\string{kr = `101'} % \comment{52 pages}
\string{kr-first-page = `103'}
\string{opal = `151'} % \comment{70 pages}
\string{opal-first-page = `153'}
\string{inter = `219'} % \comment{78 pages}
\string{inter-first-page = `221'}
\string{aggregadgets = `295'} % \comment{54 pages}
\string{aggregadgets-first-page = `297'}
\string{gadgets = `347'} % \comment{116 pages}
\string{gadgets-first-page = `349'}
\string{debug = `461'} % \comment{20 pages}
\string{debug-first-page = `463'}
\string{demos = `481'} % \comment{10 pages}
\string{demos-first-page = `483'}
\string{sampleprog = `491'} % \comment{14 pages}
\string{sampleprog-first-page = `493'}
\string{gilt = `505'} % \comment{20 pages}
\string{gilt-first-page = `507'}
\string{c32 = `525'} % \comment{12 pages}
\string{c32-first-page = `527'}
\string{lapidary = `537'} % \comment{36 pages}
\string{lapidary-first-page = `539'}
\string{hints = `573'} % \comment{8 pages}
\string{hints-first-page = `575'}
\string{GlobalIndex = `580'}

\set{page = tour-first-page}
\chapter{Introduction}
The Garnet User Interface Development Environment contains a comprehensive
set of tools that make it significantly easier to design and implement
highly-interactive, graphical, direct manipulation user interfaces.  The
lower layers of Garnet provide an object-oriented, constraint-based
graphical system that allows properties of graphical objects to be
specified in a simple, declarative manner and then maintained automatically
by the system.  The dynamic, interactive behavior of the objects can be
specified separately by attaching high-level ``interactor'' objects to the
graphics.  The higher layers of Garnet include a number of tools to allow
various parts of the interface to be specified without programming.
The primary tools are Gilt, an interface builder, and Lapidary, a tool which
helps you build gadgets and dialog boxes.

This document will help users get acquainted with the Garnet software by
leading them through a number of exercises on line.  This entire exercise
should take about an hour.  This tour assumes that the user is familiar
with Lisp, although even non-Lispers might be able to type in the
expressions verbatim and get the correct results.

Clearly, in this short tour, a great many parts of Garnet will not be
covered, so the interested reader will need to refer to other parts of this
manual for details.

\chapter{Getting Started}
\label{startlisp}

Garnet is a software package written in CommonLisp for X/11 and the Mac,
so the first thing to do is to run X/11 and lisp on your Unix machine, or
start MCL on your Mac.  At Carnegie Mellon University, the Garnet
software is available on the AFS file server.  Elsewhere, you will
have to copy the software onto your machine, and load it into your
Lisp.  See the discussion in the Overview document for an explanation
of loading Garnet and special considerations for particular machines.


\section{Typing}
\label{typing}

Many of the names in Garnet contain colons ``:'' and hyphens ``-''.  These
are part of the names and must be typed as shown.  For example,
\pr{:filling-style} is a single name, and must be typed exactly.

In this document, the text that the user types (e.g, you) is shown underlined
in the code examples.  Most of the code looks like the following:
\begin{programexample}
* \typ{(+ 3 4)}
7
\end{programexample}
The `*' is the prompt from Lisp to tell you it is ready to accept input
(your Lisp may use a different prompt).
Do not type the `*'.  Type `(+ 3 4)'.  The next line (here \pr{7})
shows what Lisp types as a response.

If you don't like to type, you might have the Appendix of this document
displayed in an editor and just copy the commands into the Lisp
window.  In X, you can use the X cut buffer (copy the lines one-by-one
into the X cut buffer, then paste them into the Lisp window);  on the
Mac, you can edit a file using the MCL editor and do the usual
copy-and-paste operations.
The Appendix contains a list of all the commands you need to
type, to make it easier to copy them.
The appendix code by itself is stored in
the file \pr{tourcommands.lisp} which is stored in the \pr{demos} source
directory (usually \pr{garnet/src/demos/tourcommands.lisp}).
{\it Note: do
not just load \pr{tourcommands}, since it will run all the demos and quickly
quit; just copy-and-paste the commands one-by-one from the file}.

\section{Garbage Collection}

Most CommonLisp implementations use a garbage collection
mechanism that occasionally interrupts
all activity until it is completed.  At various
times during your tour, Lisp will
stop and print something like the following message:
\begin{programexample}
[GC threshold exceeded with 2,593,860 bytes in use.  Commencing GC.]
\end{programexample}
You will then have to wait until it finishes and types something like:
\begin{programexample}
[GC completed with 538,556 bytes retained and 2,055,356 bytes freed.]
[GC will next occur when at least 2,538,556 bytes are in use.]
\end{programexample}
This can happen at any time, and it causes the entire system to freeze
(although the cursor will still track the mouse).  Therefore, if
nothing is responding, Lisp and Garnet may not have crashed.  Wait for
a minute and see if they come back.

\section{Errors, etc.}
It is quite common to end up in the Lisp debugger.  This might be caused by
a bug in Garnet or because you made a small typing error.  To get out of
the debugger, you will need to type the specific command for that version
of CommonLisp (\pr{q} on CMU CommonLisp, \pr{:reset} in Allegro
CommonLisp, and Command-period in MCL).  For special instructions
about the LispWorks debugger, see the section `LispWorks' in the
Overview manual.

Often, you can just try whatever you were doing again.  However, some
errors might cause Garnet or even Lisp to get messed up.  In order of
severity, you can try the following recovery strategies after leaving
the debugger:
\begin{itemize}
\item If Lisp does not seem to be responding, try typing {\tt\char`\^}C (or whatever your
break character is -- Command-comma in MCL) {\it to the lisp window}
(move the mouse cursor to the Lisp window first).

\item If you typed a line incorrectly, try typing it again the correct way.

\item If that does not work, try destroying the object you were creating and
starting over from where you first started creating the object.  To destroy
an object that you created using \pr{(create-instance 'xxx ...)}, just type
\pr{(opal:destroy xxx)}.  Note that on the \pr{create-instance} there
is a quote mark, but not on the destroy call.

\item If you were in the first part of the tour (section \ref{LearnGarnet}), then
if that does not work, try destroying the window and starting over from the
top: \pr{(opal:destroy MYWINDOW)}.  If you were in the Othello part, try typing
\pr{(stop-othello)}.

\item If that does not work, try quitting Lisp and restarting.  For CMU
CommonLisp, type \pr{(quit)} to get out of Lisp; for Lucid, type
\pr{(system:quit)}; for LispWorks, type \pr{(bye)};  for Allegro,
type \pr{:ex}; and for MCL type \pr{(quit)}.  See section
\ref{startlisp} about how to start Lisp, and section \ref{quitting}
about quitting.

\item Finally, you can always logout and log back in.
\end{itemize}

In the Appendix
of this document is a list of all the commands you are supposed
to type in.  This will be useful if you need to start over and don't want
to have to read through everything to get to where you were.  If you
are starting at the Othello part (section \ref{Othello}), you do not have
to execute any of the commands before that (except to load Garnet and the
tour).

If Lisp seems to be stuck in an infinite loop, you can break out by typing
the break character (often {\tt\char`\^}C \dash control-C) or the abort command in
MCL (Command-comma).  It will throw you into the debugger.

If you start something over, or retype a command, you may see messages
like:
\begin{programexample}
Warning - create-schema is destroying the old \#k{\tt\char`\<}MGE::TRILL{\tt\char`\>}.
\end{programexample}
This is a debugging statement is you can just ignore it.

There are a large number of debugging functions and techniques provided to
help fix Garnet toolkit code, but these are not explained in this tour.
See the debugging manual.

% \begin{comment}
% *****************************
% LOGGING IN
% *****************************
% 
% Ask for user's name and login and e-mail address.
% Automatically send bam the name in a mail message.
% 
% System will also Use-package kr, kr-debug
% 
% \end{comment}
% \begin{comment}
% *****************************
% BASICS
% *****************************
% \end{comment}
\chapter{Learning Garnet}
\label{LearnGarnet}

\section{A Note on Packages}
\index{Packages}\index{KR (Package)}\index{Opal (Package)}
\index{Inter (Package)}\index{Garnet-Gadgets (Package)}
\index{Garnet-Debug (Package)}

The Garnet software is divided into a number of Lisp packages.  A
{\it package} may be thought of as a module containing procedures and
variables that are all associated in some way.  Usually, the
programmer works in the \pr{user} package, and is not aware of
other packages in Lisp.  In Garnet, however, function calls are
frequently accompanied by the name of the package in which the function
was defined.

For example, one of the packages in Garnet is \pr{opal}, which
contains all the objects and procedures dealing with graphics.  To
reference the \pr{rectangle} object, which is defined in \pr{opal},
the user has to explicitly mention the package name, as in
\pr{opal:rectangle}.

On the other hand, the package name may be omitted if the user
calls \pr{use-package} on the package that is to be referenced.  That
is, if the command \pr{(use-package :OPAL)} or \pr{(use-package
`OPAL')} is issued, then the \pr{rectangle} object may be referenced
without naming the \pr{opal} package.

The recommended `Garnet Style' is to \pr{use-package} only one
Garnet package -- \pr{KR} -- and explicitly reference objects in other
packages.  This convention is followed in the code examples below.
The file \pr{tour.lisp} that you loaded contains the line
\pr{(use-package :KR)}, which implements this convention.  You will
probably want to put this line at the top of all your future Garnet
programs as well.

The packages in Garnet include:
\begin{itemize}
\item \pr{KR} - contains the procedures for creating and accessing objects.  This
contains the functions \pr{create-instance}, \pr{gv}, \pr{gvl},
\pr{s-value}, and \pr{o-formula}.

\item \pr{Opal} - contains the graphical objects and some functions for them.

\item \pr{Inter} - contains the interactor objects for handling the mouse.

\item \pr{Garnet-Gadgets} - (nicknamed \pr{gg}) contains a collection of predefined
`gadgets' like menus and scroll bars.

\item \pr{Garnet-Debug} - (nicknamed \pr{gd}) contains a number of debugging
functions.  These are not discussed in this tour, however.
\end{itemize}


\section{A Note on Refresh}
\index{Refreshing windows}
\index{Main-Event-Loop}
In X/11 and Mac QuickDraw, pictures drawn to windows need to be
redrawn if the window is covered and then uncovered.  Garnet handles
this automatically for you by through a background process
which detects this situation and redraws windows when necessary.
In most lisps, Garnet launches this \pr{main-event-loop} process
itself.  On the Mac, MCL runs a background process anyway, and Garnet
supplies the necessary functions that handle graphics redrawing.
This function is also responsible for processing
mouse and keyboard input to Garnet windows.

The \pr{main-event-loop} background process starts without any special
attention in most lisps, including Allegro, Lucid, CMUCL, and MCL.  If
you are running LispWorks, then there is an initialization procedure
for multiprocessing that you must perform before loading Garnet.
Please consult the `LispWorks' section of the Overview Manual, the
first section in this Garnet Reference Manual.

Unfortunately, if you are not running a recent version of Allegro,
Lucid, CMUCL, MCL, or LispWorks, your Lisp may not support background
processes. In this case, you must explicitly run
the function yourself.  If you notice that windows are not refreshing
properly after becoming uncovered (or de-iconified), or that Garnet is
completely ignoring all your keyboard and mouse input, then type the
following into Lisp:
\begin{programexample}
* \typ{(inter:main-event-loop)}
\end{programexample}
This function loops forever, so you then have to hit the
\pr{F1} key while the cursor is in a Garnet window to exit
\pr{main-event-loop}.  Alternatively, you can type {\tt\char`\^}C or Command-period,
or whatever your operating system break character is,
in the Lisp window.  Also, it is permissible (though unnecessary) to call
\pr{main-event-loop} within a version of Lisp which supports background
processes -- the function first checks if another \pr{main-event-loop}
is already running in the background, and if so, it returns immediately.


\section{Loading Garnet and the Tour}

The Overview document discusses how to load the Garnet software.  In
summary, you will load the file \pr{Garnet-Loader} and this will load all
the standard software.  After that, you need to load the special file
\pr{tour.lisp}, which is in the \pr{src/demos} sub-directory.
For example, if the Garnet files are in the directory
\pr{/usr/xxx/garnet/}, then type the following:

\begin{programexample}
* \typ{(load `/usr/xxx/garnet/garnet-loader')}
\end{programexample}
Which will print out lots of stuff.  Then type:
\begin{programexample}
* \typ{(garnet-load `demos:tour')}
\end{programexample}
Note that \pr{garnet-load}\index{Garnet-Load} is a useful procedure
provided by Garnet to simplify loading Garnet files.  It takes one
argument (in this case \pr{`demos:tour'}), a two-part string consisting of
the a Garnet subdirectory reference (eg, \pr{`demos'}) and
the name of a file (eg, `tour'), separated by a colon.  The procedure
searches the directory associated with that package for a Lisp file (either
compiled or uncompiled) of that name.



\section{Basic Objects}
Now you are going to start creating some Garnet Toolkit objects.

\index{Create-instance}
Garnet is an object-oriented system, and you create objects using the
function \pr{create-instance}, which takes a quoted name for the new
object, the type of object to create, and then some other optional
parameters.  First, you will create a window object.

\index{interactor-window}
Type the text shown underlined to Lisp.  Be sure to start with an open
parenthesis and be careful about where the quotes and colons go.
\begin{programexample}
* \typ{(create-instance 'MYWINDOW inter:interactor-window)}
\#k{\tt\char`\<}MYWINDOW{\tt\char`\>}
\end{programexample}

\index{update}
You won't see anything yet, because Garnet waits for an \pr{update} call
before showing the results.  Now type:
\begin{programexample}
* \typ{(opal:update MYWINDOW)}
\end{programexample}
and the window should appear.

You can move the window around and change its size just like any other
X or Mac window, in whatever way you have your X window manager set up
to do this.

\index{s-value}
\index{aggregate}
Now, you are going to create an ``aggregate'' object to hold all the other
objects you create.  An aggregate holds a collection of other objects; it
does not have any graphic appearance itself.
\begin{programexample}
* \typ{(create-instance 'MYAGG opal:aggregate)}
\#k{\tt\char`\<}MYAGG{\tt\char`\>}
\end{programexample}
This aggregate will be the special top level aggregate in the window, that
will hold all the objects to be displayed in the window.  You will use the
function \pr{s-value} which sets the value of a ``slot'' (also called an
instance variable) of the object.  \pr{S-value} takes the object, the slot
and the new value.  To read the value of the slot, use the function \pr{gv},
which stants for ``get value''.  All slot names in Garnet start
with a colon.
\begin{programexample}
* \typ{(s-value MYWINDOW :aggregate MYAGG)}
\#k{\tt\char`\<}MYAGG{\tt\char`\>}
* \typ{(gv MYWINDOW :aggregate)}
\#k{\tt\char`\<}MYAGG{\tt\char`\>}
\end{programexample}

\index{moving-rectangle}
\index{rectangle}
Now, you will create a rectangle.
\begin{programexample}
* \typ{(create-instance 'MYRECT MOVING-RECTANGLE)}
\#k{\tt\char`\<}MYRECT{\tt\char`\>}
\end{programexample}
\sm{[Note: MOVING-RECTANGLE is defined in the \pr{user} package by
\pr{tour.lisp} as a specialization of the general \pr{opal:rectangle} prototype.]}

\index{add-component}
Again, this is not visible yet.  First, the rectangle must be added to the
aggregate, and then the update procedure must be called.  Adding the
rectangle uses the function
\pr{add-component} which takes the aggregate and the new object to add to
it.
\begin{programexample}
* \typ{(opal:add-component MYAGG MYRECT)}
\#k{\tt\char`\<}MYRECT{\tt\char`\>}
* \typ{(opal:update MYWINDOW)}
NIL
\end{programexample}

The rectangle should now appear in the window.

\index{filling-style}
\index{gray-fill}
All objects have a number of properties,
such as their position, size and color.  So far, all the objects have used
the default values for properties.  You will now change the color of the
rectangle by setting its \pr{:filling-style} slot.  Remember that slot
names begin with a colon, and that nothing happens until you do the
\pr{update}.
\begin{programexample}
* \typ{(s-value MYRECT :filling-style opal:gray-fill)}
\#k{\tt\char`\<}GRAY-FILL{\tt\char`\>}
* \typ{(opal:update MYWINDOW)}
NIL
\end{programexample}

The other filling styles that are available include \pr{opal:light-gray-fill,
opal:dark-gray-fill, opal:black-fill, opal:white-fill}, and
\pr{opal:diamond-fill}.  These are all ``halftone'' shades, which
means that they are created by turning some pixels on and others off.
If you have a color screen, you might also try \pr{opal:red-fill,
opal:blue-fill, opal:green-fill, opal:yellow-fill, opal:purple-fill}, etc.

Now, you will create a text object.  Here, for the first time, you will
supply some extra values for slots when the object is created, rather than
just using \pr{s-value} afterward.  Objects have a large number of slots
and the ones that are not specified use the default values.
To specify a slot at creation time, each name and value is
enclosed in a separate parenthesis pair.  Note that you can type carriage
return where-ever you want.  After the text is created, add it to the
aggregate and update the window.
\index{Hello World}
\index{Cursor-Multi-Text}
\begin{programexample}
* \typ{(create-instance 'MYTEXT opal:text (:left 200)(:top 80)
      (:string `Hello World'))}
\#k{\tt\char`\<}MYTEXT{\tt\char`\>}
* \typ{(opal:add-component MYAGG MYTEXT)}
\#k{\tt\char`\<}MYTEXT{\tt\char`\>}
* \typ{(opal:update MYWINDOW)}
NIL
\end{programexample}
The \pr{:top} of the string is just its \pr{Y} value, and the \pr{:left} is
just the \pr{X} value, and they are, of course, independent.

You can change the position (\pr{:left} and \pr{:top}) and string of
MYTEXT using \pr{s-value} if you want, like the following:
\begin{programexample}
* \typ{(s-value MYTEXT :top 40)}
40
* \typ{(opal:update MYWINDOW)}
NIL
\end{programexample}


\section{Formulas}
\index{formula}
\index{o-formula}
\index{gv}
An important property of Garnet is that properties of objects can be
connected using {\it constraints}.  A constraint is a relationship that is
defined once and maintained automatically by the system.  You
will constrain the string to stay at the top of the rectangle.  Then, when
the rectangle is moved, the string will move automatically.

Constraints in Garnet are expressed as {\it formulas} which are put into the
slots of objects.  Any slot can either have a value in it (like a number or
a string) or a formula which computes the value.  The formula can be an
arbitrary Lisp expression which must be passed to the Garnet function
\pr{o-formula}.  References to other objects in formulas must take a special
form.  To get the slot \pr{slot-name} from the object
\pr{other-object}, use the form \pr{(gv other-object slot-name)}, where
``gv'' stands for ``get value.''  The \pr{gv} function can be used either
inside or outside of formulas.  When used from inside a formula, \pr{gv} will
establish a dependency on the referenced slot, causing the formula to
reevaluate if the value in the referenced slot ever changes.

Now, set the top of the string to be a formula that depends on the
top of the rectangle.

Note that the particular number returned by the \pr{s-value} call will not
be the same as shown below.
\begin{programexample}
* \typ{(s-value MYTEXT :top (o-formula (gv MYRECT :top)))}
\#k{\tt\char`\<}F3875{\tt\char`\>}   {\it the number will be different}
* \typ{(opal:update MYWINDOW)}
NIL
\end{programexample}

After the update, the string should move to be at the top of the rectangle.
If you change the top of the rectangle, {\it both} the rectangle and the string
will now move:
\begin{programexample}
* \typ{(s-value MYRECT :top 50)}
50
* \typ{(opal:update MYWINDOW)}
NIL
\end{programexample}

If you want to experiment with writing your own formulas, the Lisp
arithmetic operators include \pr{+, -, floor} (for divide), and \pr{*} (for
multiply) and they must be in fully parenthesized expressions, as in
\pr{(o-formula (+ (gv MYRECT :top) 7))}.
To get the width and height of an object from inside a
formula, use \pr{(gv {\it obj} :width)} and \pr{(gv {\it obj}
:height)}.  You could try, for example, to get the text to stay centered in
X (\pr{:left}) and Y (\pr{:top}) inside the rectangle.


\section{Interaction}

Now, you will get the objects to respond to input.  To do this, you attach
an {\it interactor} to the object.  Interactors handle the mouse and keyboard
and update graphical objects.

\index{Move-Grow-Interactor}
First, you will have the rectangle move with the mouse.  To do this, you
create a \pr{move-grow-interactor} and tell it to operate on MYRECT.
The interactor will start whenever the mouse is pressed \pr{:in MYRECT},
and the interactor works in MYWINDOW.  The interactor will continue to
run no matter where the mouse is moved while the button is held down.

It is not necessary to call
\pr{update} to get interactors to start working; they start as soon as
they are created.  However, if you are not using a recent version of
CMU, Allegro, LispWorks, Lucid, or MCL CommonLisp, interactors only run
while the \pr{main-event-loop} procedure is operating.
\pr{Main-Event-Loop} does not exit, so you will have to hit the \pr{F1} key
while the cursor is in the Garnet window, or type {\tt\char`\^}C (or
whatever your operating system break character is) while the cursor is in
the Lisp window, to be able to type further Lisp expressions.

\begin{programexample}
* \typ{(create-instance 'MYMOVER inter:move-grow-interactor
	(:start-where (list :in MYRECT))
	(:window MYWINDOW))}
\#k{\tt\char`\<}MYMOVER{\tt\char`\>}
\end{programexample}
If your Lisp requires it, then type:
\begin{programexample}
* \typ{(inter:main-event-loop)}
\end{programexample}

Now you can press with the left button over the rectangle, and while the
button is held down, move the rectangle around.  (The first time you press
on the rectangle, it may take a while, as Lisp swaps in the appropriate code.)
Notice that the text
string moves up and down also.  The text string does not move left and
right, however, since there is no constraint on the \pr{:left} of the
string, only on the \pr{:top} (unless you have written some extra formulas
other than the one described above).

A different interactor allows you to type into text strings.  This is called a
\pr{text-interactor}.  The code below will cause the text interactor to start
when you press the right mouse button, and stop when you press the right mouse
button again.  This will allow you to type carriage returns into the string
and to move the cursor point by hitting the left button inside the string.
(Before typing these commands, hit the F1 key to exit  \pr{main-event-loop}
if necessary).
\index{Text-Interactor}
\begin{programexample}
* \typ{(create-instance 'MYTYPER inter:text-interactor
	(:start-where (list :in MYTEXT))
	(:window MYWINDOW)
	(:start-event :rightdown)
	(:stop-event :rightdown))}
\#k{\tt\char`\<}MYTYPER{\tt\char`\>}
\end{programexample}
If your Lisp requires it, then type:
\begin{programexample}
* \typ{(inter:main-event-loop)}
\end{programexample}

Now, if you press with the right mouse button on the string, you can change
the string by typing.  The available editing commands include:
\begin{description}
\item[] \pr{{\tt\char`\^}h, delete, backspace}: delete previous character.

\item[] \pr{{\tt\char`\^}w, {\tt\char`\^}backspace, {\tt\char`\^}delete}: delete previous word.

\item[] \pr{{\tt\char`\^}d}: delete next character.

\item[] \pr{{\tt\char`\^}u}: delete entire string.

\item[] \pr{{\tt\char`\^}b, left-arrow}: go back one character.

\item[] \pr{{\tt\char`\^}f, right-arrow}: go forward one character.

\item[] \pr{{\tt\char`\^}n, down-arrow}: go vertically down one line.

\item[] \pr{{\tt\char`\^}p, up-arrow}: go vertically up one line.

\item[] \pr{{\tt\char`\^}{\tt\char`\<}, {\tt\char`\^}comma, home}: go to the beginning of the string.

\item[] \pr{{\tt\char`\^}>, {\tt\char`\^}period, end}: go to the end of the string.

\item[] \pr{{\tt\char`\^}a}: go to beginning of the current line.

\item[] \pr{{\tt\char`\^}e}: go to end of the current line.

\item[] \pr{{\tt\char`\^}y, insert}: insert the contents of the X or Mac cut buffer into
the string at the current point.

\item[] \pr{{\tt\char`\^}c}: copy the current string to the X or Mac cut buffer.

\item[] \pr{enter, return, {\tt\char`\^}j, {\tt\char`\^}J}: Go to new line.

\item[] \pr{left button down inside the string}: move the cursor to the
specified point.

\item[] \pr{{\tt\char`\^}G}: Abort the edits and return the string to the way it was before
editing started.
\end{description}
All other characters go into the string (except other control characters
which beep).  You can also move the cursor with the mouse by clicking in
the string.

(In X, to type to a window, the mouse cursor must be inside the window, so to
type to the ``Hello World'' string, the mouse cursor must be inside the Garnet
window, and to type to Lisp, the cursor should be inside the Lisp
window.  On the Mac, you have to click the mouse on the title-bar of
the window you want to type into, so you will have to click
alternately on the Garnet window and the lisp listener.)


If you make the text string be multiple lines, by typing a carriage
return into it,
then you can control whether the lines are centered, left or right
justified.  This is controlled by the \pr{:justification} slot of
MYTEXT, which can be \pr{:left, :center}, or \pr{:right}.
(Before typing these commands, hit the F1 key to exit  \pr{main-event-loop}
if necessary).
\index{justification}
\begin{programexample}
* \typ{(s-value MYTEXT :justification :right)}
:RIGHT
* \typ{(opal:update MYWINDOW)}
NIL
* \typ{(s-value MYTEXT :justification :center)}
:CENTER
* \typ{(opal:update MYWINDOW)}
NIL
\end{programexample}

Of course, you can type to the string while it is centered or
right-justified, and you can move around the rectangle with the mouse and
the string will still follow.

\section{Higher-level Objects}

Now, you are going to create instances of pre-created objects from the
``Garnet Gadget Set.''  The Gadget Set contains a large collection of menus,
buttons, scroll bars, sliders, and other useful {\it interaction techniques}
(also called ``widgets'').  You will be using a set of ``radio buttons''
and a slider.

First, however, you should make the window bigger (in whatever way you
do this in
your window manager).

\subsection{Buttons}

First, you will create a set of 3 ``radio'' buttons that will determine
whether the text
is centered, left, or right justified.  The parameter that tells the
buttons what the labels should be is called \pr{:Items}.  This slot is
passed a quoted list.  The radio buttons will appear at the right of
the string.
\index{Radio-Button-Panel}
\begin{programexample}
* \typ{(create-instance 'MYBUTTONS gg:radio-button-panel
        (:items '(:center :left :right))
	(:left 350)(:top 20))}
\#k{\tt\char`\<}MYBUTTONS{\tt\char`\>}
* \typ{(opal:add-component MYAGG MYBUTTONS)}
\#k{\tt\char`\<}MYBUTTONS{\tt\char`\>}
* \typ{(opal:update MYWINDOW)}
NIL
\end{programexample}

If your Lisp requires it, then type:
\begin{programexample}
* \typ{(inter:main-event-loop)}
\end{programexample}

Now, you can click on the radio buttons with the left mouse button, and the
dot will move to whichever one you click on.

\index{Value Slot}
Next, you will use a constraint to tie the value of the \pr{:justification}
field of the text object to the value of the radio buttons.  The current
value of the radio buttons is conveniently kept in the \pr{:value} field.
(Before typing these commands, hit the F1 key to exit  \pr{main-event-loop}
if necessary).

\begin{programexample}
* \typ{(s-value MYTEXT :justification (o-formula (gv MYBUTTONS :value)))}
\#k{\tt\char`\<}F2312{\tt\char`\>}   {\it the number will be different}
* \typ{(opal:update MYWINDOW)}
NIL
\end{programexample}
If your Lisp requires it, then type:
\begin{programexample}
* \typ{(inter:main-event-loop)}
\end{programexample}

Now, whenever you press on one of the buttons, the text will re-adjust
itself.

All of the built-in toolkit items have a large number of parameters to
allow users to customize their look and feel.  For example, you can change
the radio buttons to be horizontal instead of vertical:
(From now on, you will have to remember to hit the F1 key to exit
\pr{main-event-loop} if necessary before typing commands without these
reminders).
\begin{programexample}
* \typ{(s-value MYBUTTONS :direction :horizontal)}
:HORIZONTAL
* \typ{(opal:update MYWINDOW)}
NIL
\end{programexample}

Now, change it back to be vertical:
\begin{programexample}
* \typ{(s-value MYBUTTONS :direction :vertical)}
:VERTICAL
* \typ{(opal:update MYWINDOW)}
NIL
\end{programexample}

\subsection{Slider}

Next, you will do a similar thing to get the gray shade of the
rectangle to be attached to an on-screen slider.  First, create a Garnet
vertical slider object:
\index{V-Slider}
\begin{programexample}
* \typ{(create-instance 'MYSLIDER gg:v-slider
	(:left 10)(:top 20))}
\#k{\tt\char`\<}MYSLIDER{\tt\char`\>}
* \typ{(opal:add-component MYAGG MYSLIDER)}
\#k{\tt\char`\<}MYSLIDER{\tt\char`\>}
* \typ{(opal:update MYWINDOW)}
NIL
\end{programexample}
If your Lisp requires it, then type:
\begin{programexample}
* \typ{(inter:main-event-loop)}
\end{programexample}

This slider can be operated in a number of ways, all using the left mouse
button.  Press on the top arrow to move up one unit, and the down arrow to
move down one.  The double arrow buttons move up and down by five (the
increment amount can be changed by using \pr{s-value} on the
\pr{:scr-incr} and \pr{:page-incr} slots of MYSLIDER).  You can
also press on
the black indicator arrow and drag it to a new position.  Finally. you can
press in the top number area, then type a new number value, and then hit
carriage return.

Of course the value returned by the slider does not affect anything yet.
To change the color of the rectangle, you will use the Garnet function
\pr{Halftone}, which takes a number from 0 to 100 and returns a
\pr{:filling-style} that is that percentage black.  Connect the filling
style of the rectangle to the value returned by the slider:

\begin{programexample}
* \typ{(s-value MYRECT :filling-style
		(o-formula (opal:halftone (gv MYSLIDER :value))))}
\#k{\tt\char`\<}F5940{\tt\char`\>}   {\it the number will be different}
* \typ{(opal:update MYWINDOW)}
NIL
\end{programexample}
If your Lisp requires it, then type:
\begin{programexample}
* \typ{(inter:main-event-loop)}
\end{programexample}

Now when you change the value of the slider, the color of the rectangle
will change.  Note that halftone only can generate 17 different gray
colors, so a range of numbers for the slider will generate the same color.

\chapter{Playing Othello}
\label{Othello}
\index{Othello}
\index{start-othello}

Now you can play the Othello game we created using the Garnet Toolkit.

To bring up the game, type:
\begin{programexample}
* \typ{(start-othello)}
T
\end{programexample}
The game board will appear on the
screen.  There are various things you can control in the game.  You can put
new pieces down on the board by just pressing with the left mouse button.
In Othello, you can put a piece in a position where you are next to the
other player's marker, and one of your markers is in a straight line from
where you are going to play.
If you try to place your marker in an illegal place, the
game will beep.  This game does not try to play against you; you must
handle both players (or get someone else to play with you).
If a player does not want to move (or has no legal moves), then the
``Pass'' menu item can be selected.  This implementation does not detect
when the game is over.  The current score (which is the number of squares
that the player controls) is shown in the top left box.

To start over, press on the menu button marked ``Start.''  This will start
a new game with a board that has the number of squares shown by the scroll
bar.  The default is 8 by 8.  To change the scroll bar value, press on the
arrows.  (Changing the scroll bar does not change the current board; it
takes affect the next time you hit ``Start'' from the menu.)

``Stop'' just erases the board, and ``Quit'' exits the game.  (You don't
have to quit before going on to the next section.)

\chapter{Modifying Othello}

We created an editor that allows you to change what the Othello playing
pieces look like.  This is editor is just a small toy program that was
created quickly by David Kosbie in the Garnet group especially for this tour.

If you quit out of the Othello game, bring it back up using \pr{(start-othello)}.

Othello has a tall window on the left side of the screen containing
the current 2 Othello playing  pieces at the top: a white
and a black circle.  Underneath is a command button (``Delete'') and 3
menus.  The top left menu is for different types of objects: rectangles,
rounded rectangles, circles and ovals.  The bottom left menu is for line
styles (the way the outlines of objects are drawn): no outline, dotted
outline, thin, thicker or very thick outline.  The menu on the right is for
how the inside of objects looks: no filling inside, white, grey, black or
various patterns.

Press with the left mouse button over any of the menus to change the
current mode.

To draw a new object in either playing
piece, just use the {\it right} mouse button
to drag out the dimensions for the new object.  Press down the right button
inside whichever piece you want to modify where you want one corner of the
new object to be, move the cursor while holding down, and
release at the other corner.  The type, line styles, and
inside of the new object come from the current values of the menus.

Objects can be selected by pressing over them with the {\it left} mouse
button.  (Some objects require that you press on the edge (border) of the
object, and others allow you to press anywhere inside.)
When an object is selected, 12 small boxes are shown on the borders of
the object.  (The small boxes are on the bounding rectangle of the object, which
may be a little confusing for circles.)
The black boxes can be used to change the object's size, and
the white boxes are used to move the object.  Just press with the left
button over one of the boxes, and then adjust the size or position while
holding down.  The editor will not let you move or grow an object so that
it goes outside the game piece area.

The selected object can also be deleted or changed.  Delete it by just
hitting the Delete button in the menu when the object is selected.  If you
press on a new line style or
filling style while an object is selected, the object's outline and color will
change.  (You can't change an object's type.)  Note that as you select
objects, the menus change to show the object's current styles.

Every time you edit one of the playing pieces, the Othello game display
also changes to reflect the edits.  This is handled automatically by Garnet
using inheritance.


\chapter{Using GarnetDraw}
\index{garnetdraw}

There a useful utility called \pr{GarnetDraw} which is a
relatively simple drawing program written using Garnet.  Using this
application, you can draw pictures with many of the basic Garnet objects
(like circles, rectangles, and lines), and then save the picture to a file.
Since the file format for storing the created objects is simply a Lisp file
which creates aggregadgets, you might be able to use GarnetDraw to prototype
application objects (but Lapidary is probably better for this).

GarnetDraw uses many sophisticated features of Garnet including gridding,
PostScript printing, selection of all objects in a region, moving and growing
of multiple objects, menubars, and the \pr{save-gadget} and \pr{load-gadget}
dialog boxes.

To load and start GarnetDraw, type:
\begin{programexample}
* \typ{(garnet-load `demos:garnetdraw')}

* \typ{(garnetdraw:do-go)}
\end{programexample}

GarnetDraw works like most Garnet programs: select in the palette
with any button, draw in the main window with the right button, and select
objects with the left button.  Select multiple objects with shift-left or
the middle mouse button.  Change the size of objects by pressing on black
handles and move them by pressing on
white handles.  The line style and color and filling color can be
changed for the selected object and for further drawing by clicking on
the icons at the bottom of the palette.

You might want to save a picture to a file, and then bring the file up in
your editor to see the kind of code that GarnetDraw generates.  There should
be a top-level aggregadget that has your drawn objects as components.

To quit GarnetDraw, either select `Quit' from the menubar, or type:

\begin{programexample}
* \typ{(garnetdraw:do-stop)}
\end{programexample}



\chapter{Cleanup}
\label{quitting}

If you are not in a Lisp which supports background processes, and
you are running something in Garnet, then you need to type F1 in a
Garnet window or {\tt\char`\^}C in your Lisp window to get back to the Lisp
read-eval-print loop.

\index{stop-tour}
To get rid everything at once (MYWINDOW, the Othello game, and the
editor for the game pieces), just type:
\begin{programexample}
* \typ{(stop-tour)}
`Thank you for your interest in the Garnet Project'
\end{programexample}

\index{stop-othello}
Otherwise,
to just get rid of Othello and the editor, you can hit on the ``Quit'' menu
button or type \pr{(stop-othello)} to Lisp.  To just get rid of MYWINDOW,
type \pr{(opal:destroy MYWINDOW)}.

The command that exits Lisp is different for different implementations.
For CMU CommonLisp, type: \programexample{* \typ{(quit)}}

for Lucid CommonLisp, type: \programexample{* \typ{(system:quit)}}

for LispWorks, type: \programexample{* \typ{(bye)}}

for Allegro CommonLisp, type: \programexample{* \typ{:ex}}

and for MCL, type: \programexample{* \typ{(quit)}}

This returns you to the shell (or to the finder on the Mac), and you
can log out.  It is not necessary to run \pr{(stop-othello)} or
\pr{(stop-tour)} before quitting Lisp.

If the quit command doesn't work for any reason,
you can probably quit by typing {\tt\char`\^}Z to pause to the shell and
then kill the lisp process (or just log out).

\chapter{Conclusion}
We hope you have enjoyed your tour through Garnet.  There are, of course,
many features and capabilities that have not been demonstrated.  These are
described fully in the various manuals and papers about the Garnet project
and its parts.  The next step might be to run the Gilt interface
builder, since it does not require that you learn much about how
Garnet works.  See the Gilt manual.

\chapter*{Appendix: List of commands}
This appendix lists all the commands that the tour has you type.  This is
useful as a quick reference if you need to restart due to an error.
These commands are stored in
the file \pr{tourcommands.lisp} which is stored in the \pr{demos} source
directory (usually \pr{garnet/src/demos/tourcommands.lisp}).
If you have this document in a window on the screen, you can
copy-@{\tt\char`\|}and-@{\tt\char`\|}paste to move text from below into your Lisp window.
{\it Note: do
not just load \pr{tourcommands}, since it will run all the demos and quickly
quit; just copy the commands one-by-one from the file}.

This listing does not show the prompts or Lisp's responses to these commands.

{\bf First, load the Garnet software.  You will have to replace \pr{xxx}
with your directory path to Garnet:}
\begin{programexample}
(load `/xxx/garnet/garnet-loader')
(garnet-load `demos:tour')
\end{programexample}


{\bf Start here after Garnet and the tour software is loaded:}
\begin{programexample}
(create-instance 'MYWINDOW inter:interactor-window)
(opal:update MYWINDOW)

(create-instance 'MYAGG opal:aggregate)
(s-value MYWINDOW :aggregate MYAGG)
(gv MYWINDOW :aggregate)
(create-instance 'MYRECT MOVING-RECTANGLE) {\it ; In the USER package}
(opal:add-component MYAGG MYRECT)
(opal:update MYWINDOW)

(s-value MYRECT :filling-style opal:gray-fill)
(opal:update MYWINDOW)

(create-instance 'MYTEXT opal:text (:left 200)(:top 80)
	(:string `Hello World'))
(opal:add-component MYAGG MYTEXT)
(opal:update MYWINDOW)

(s-value MYTEXT :top 40)
(opal:update MYWINDOW)

(s-value MYTEXT :top (o-formula (gv MYRECT :top)))
(opal:update MYWINDOW)

(s-value MYRECT :top 50)
(opal:update MYWINDOW)

(create-instance 'MYMOVER inter:move-grow-interactor
  (:start-where (list :in MYRECT))
  (:window MYWINDOW))
\#-(or cmu allegro lucid lispworks apple)  {\it ;only do this if your Lisp is NOT a recent}
(inter:main-event-loop)                   {\it ;version of CMU, Allegro, Lucid, or LispWorks}
                                          {\it ;type F1 or {\tt\char`\^}C to exit when finished.}

(create-instance 'MYTYPER inter:text-interactor
  (:start-where (list :in MYTEXT))
  (:window MYWINDOW)
  (:start-event :rightdown)
  (:stop-event :rightdown))
\#-(or cmu allegro lucid lispworks apple)  {\it ;only do this if your Lisp is NOT a recent}
(inter:main-event-loop)                   {\it ;version of CMU, Allegro, Lucid, or LispWorks}
                                          {\it ;type F1 or {\tt\char`\^}C to exit when finished.}

(s-value MYTEXT :justification :right)
(opal:update MYWINDOW)

(s-value MYTEXT :justification :center)
(opal:update MYWINDOW)

(create-instance 'MYBUTTONS gg:radio-button-panel
  (:items '(:center :left :right))
  (:left 350)(:top 20))
(opal:add-component MYAGG MYBUTTONS)
(opal:update MYWINDOW)
\#-(or cmu allegro lucid lispworks apple)  {\it ;only do this if your Lisp is NOT a recent}
(inter:main-event-loop)                   {\it ;version of CMU, Allegro, Lucid, or LispWorks}
                                          {\it ;type F1 or {\tt\char`\^}C to exit when finished.}

(s-value MYTEXT :justification (o-formula (gv MYBUTTONS :value)))
(opal:update MYWINDOW)
\#-(or cmu allegro lucid lispworks apple)  {\it ;only do this if your Lisp is NOT a recent}
(inter:main-event-loop)                   {\it ;version of CMU, Allegro, Lucid, or LispWorks}
                                          {\it ;type F1 or {\tt\char`\^}C to exit when finished.}

(s-value MYBUTTONS :direction :horizontal)
(opal:update MYWINDOW)

(s-value MYBUTTONS :direction :vertical)
(opal:update MYWINDOW)

(create-instance 'MYSLIDER gg:v-slider
  (:left 10)(:top 20))
(opal:add-component MYAGG MYSLIDER)
(opal:update MYWINDOW)
\#-(or cmu allegro lucid lispworks apple)  {\it ;only do this if your Lisp is NOT a recent}
(inter:main-event-loop)                   {\it ;version of CMU, Allegro, Lucid, or LispWorks}
                                          {\it ;type F1 or {\tt\char`\^}C to exit when finished.}

(s-value MYRECT :filling-style (o-formula
				(opal:halftone (gv MYSLIDER :value))))
(opal:update MYWINDOW)
\#-(or cmu allegro lucid lispworks apple)  {\it ;only do this if your Lisp is NOT a recent}
(inter:main-event-loop)                   {\it ;version of CMU, Allegro, Lucid, or LispWorks}
                                          {\it ;type F1 or {\tt\char`\^}C to exit when finished.}
\end{programexample}

{\bf To just get Othello to run, execute the following line.
You do not have to enter any of the previous code to run Othello and
the editior (except for the software loading, of course).}
\begin{programexample}

(start-othello)

\end{programexample}


{\bf To just load and run GarnetDraw, execute the following lines.}

\begin{programexample}

(garnet-load `demos:garnetdraw')
(garnetdraw:do-go)

\end{programexample}

{\bf Cleaning up and quitting:}
\begin{programexample}
{\it ;;; * To quit all editors and demos and destroy all windows}

(stop-tour)
(garnetdraw:do-stop)   {\it ; if running}

{\it ;;; * To leave lisp}

\#+cmu   (quit)         {\it ; in CMU CommonLisp}
\#+lucid (system:quit)  {\it ; in Lucid CommonLisp}
\#+allegro :ex          {\it ; in Allegro CommonLisp}
\#+lispworks (bye)      {\it ; in LispWorks CommonLisp}
\#+apple (quit)         {\it ; in MCL}

\end{programexample}


% \begin{comment}
% 
% **********************************************************
% Later, have another section on more details of objects, etc.
% **********************************************************
% 
% 
% **********************************************************
% CREATING A GRAPHICS EDITOR (Optional)
% **********************************************************
% 
% 		left button moves objects
% 		right button edits text
% 		shift-left creates a new object
% 		shift-right deletes object under mouse
% 		scroll bar and menu as before for specifying props of next
% 			new object
% 
% create another window
% create an aggregadget of a rectangle, line, and editable-string as a prototype
% ** or use predefined one; would prefer to do this part using Lapidary****
% 
% create an aggregate to hold the objects
% 
% create a new version of the scroll bar and x-boxes to in another new window
% 
% define the create and delete functions
% 
% create all the interactors
% 
% 
% \end{comment}
% \begin{comment}
% \chapter*{References}
% \bibliography
% \end{comment}
