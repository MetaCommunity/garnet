\device{postscript}
\make{manual}
\disable{figurecontents}
\libraryfile{Garnet}
\string{TitleString = `Gilt'}
\use{Bibliography = `garnet.bib'}
\modify{FigureCounter, within unnumbered}
\modify{captionenv, fill, Spaces=Compact, below=0}

\define{programinlist=programexample, LeftMargin +0.5inch, RightMargin 0,
	above 0, below 0}


\begin{titlepage}
\begin{titlebox}
\vspace{0.6 inch}
\bg{Gilt Reference Manual:
A Simple Interface Builder for Garnet}


{\bf Brad A. Myers}
\vspace{0.3 line}
\value{date}
\end{titlebox}
\vspace{0.5 inch}
\begin{center}
{\bf Abstract}\end{center}
\begin{text}
Gilt is a simple interface layout tool that helps the user design dialog
boxes.  It allows the user to place pre-defined Garnet gadgets in a window
and then save them to a file.  There are two versions: one for Garnet
look-and-feel gadgets and one for Motif look-and-feel gadgets.
\vspace{0.5 inch}
\begin{center}
Copyright \j{w} 1990 - Carnegie Mellon University\end{center}

This research was sponsored by the Defense Advanced Research Projects
Agency (DOD), ARPA Order No. 4976, Amendment 20, under contract
F33615-87-C-1499,
monitored by the Avionics Laboratory, Air Force Wright Aeronautical
Laboratories, Aeronautical Systems Division (AFSC), Wright-Patterson AFB,
Ohio 45433-6543.

The views and conclusions contained in this document are
those of the authors and should not be interpreted as representing the
official policies, either expressed or implied, of the Defense Advanced
Research Projects Agency or the US Government.

\end{text}
\end{titlepage}



\string{overview = `1'} % \comment{26 pages}
\string{overview-first-page = `3'}
\string{apps = `27'} % \comment{12 pages}
\string{apps-first-page = `29'}
\string{tour = `41'} % \comment{20 pages}
\string{tour-first-page = `43'}
\string{tutorial = `61'} % \comment{42 pages}
\string{tutorial-first-page = `63'}
\string{kr = `101'} % \comment{52 pages}
\string{kr-first-page = `103'}
\string{opal = `151'} % \comment{70 pages}
\string{opal-first-page = `153'}
\string{inter = `219'} % \comment{78 pages}
\string{inter-first-page = `221'}
\string{aggregadgets = `295'} % \comment{54 pages}
\string{aggregadgets-first-page = `297'}
\string{gadgets = `347'} % \comment{116 pages}
\string{gadgets-first-page = `349'}
\string{debug = `461'} % \comment{20 pages}
\string{debug-first-page = `463'}
\string{demos = `481'} % \comment{10 pages}
\string{demos-first-page = `483'}
\string{sampleprog = `491'} % \comment{14 pages}
\string{sampleprog-first-page = `493'}
\string{gilt = `505'} % \comment{20 pages}
\string{gilt-first-page = `507'}
\string{c32 = `525'} % \comment{12 pages}
\string{c32-first-page = `527'}
\string{lapidary = `537'} % \comment{36 pages}
\string{lapidary-first-page = `539'}
\string{hints = `573'} % \comment{8 pages}
\string{hints-first-page = `575'}
\string{GlobalIndex = `580'}

\set{page = gilt-first-page}

\chapter{Gilt}

\section{Introduction}

\index{Gilt}
This document is the reference manual for the {\it Gilt} tool, which
is part of the Garnet User Interface Development System \cite{GarnetIEEE}.
Gilt stands for the {\it G}arnet {\it I}nterface {\it L}ayout {\it T}ool, and is
a simple interface builder for constructing dialog boxes.  A dialog box is
a collection of {\it gadgets}, such as menus, scroll bars, sliders, etc.
Gilt supplies a window containing many of the built-in Garnet gadgets (see
Figure \ref{gadgetwindow}), from
which the user can select the desired gadgets and place them in the work
window.  Gilt does {\it not} allow constraints to be placed on objects or for
new gadgets or application-specific objects to be created.

There are two sets of gadgets in Gilt.  Each allows you to create dialog boxes
with a consistent look-and-feel.  The standard Garnet gadgets are shown in
Figure \ref{gadgetwindow}, and the Motif style gadgets are in Figure
\ref{motifgadgetwindow}).  Both versions operate the same way.  You can
toggle between the standard and Motif gadget palettes by selecting
`\pr{Load Other Gadgets}' from the main Gilt menubar.



\begin{figure}
\bar{}
\begin{center}
\graphic{Postscript=`gilt/giltgarnetgadgets.PS', boundingbox=file,
magnify=.75}\end{center}
\caption{The Gilt gadget window for the Garnet look and feel.  All of
the gadgets that can be put into the window are shown.  The check
boxes are selected.}
\tag{GadgetWindow}
\bar{}
\end{figure}

\begin{figure}
\begin{center}
\graphic{Postscript=`gilt/giltmotifgadgets.PS',
	magnify=.75, boundingbox=file}\end{center}
\caption{The Gilt gadget window for the Motif look and feel.}
\tag{motifgadgetwindow}
\bar{}
\end{figure}

There is a more powerful interactive design tool in Garnet called Lapidary
\cite{garnetLapidary}.  Lapidary allows new gadgets to be constructed
from scratch, and allows application-specific graphics to be created
without programming.  However, Lapidary does not support the placement of
the existing Garnet gadgets.

\section{Loading Gilt}

Gilt is {\it not} automatically loaded when you load Garnet.  After Garnet
is loaded, to load Gilt do:
\index{Garnet-Gilt-Loader}
\begin{programexample}
(load Garnet-Gilt-Loader)
\end{programexample}

There is only one version of Gilt, but you can specify what set of gadgets
should appear in the palette when the windows appear.  This is determined
by a required parameter to \pr{do-go}.  To start Gilt, do:

\index{Starting Gilt}
\index{Do-Go (Gilt)}
\begin{programexample}
(gilt:do-go :motif)
{\rm or}
(gilt:do-go :garnet)
\end{programexample}

\index{Stopping Gilt}
\index{Quitting Gilt}
\index{Do-Stop (Gilt)}
Gilt can be stopped by selecting `\pr{Quit}' from the menubar, or by executing
\pr{(gilt:do-stop)}.


\section{User Interface}

Gilt displays three windows: The gadgets window, the main command
window, and the work window.  The main command window is shown in Figure
\ref{commandwindow}.  Figure \ref{workwindow} shows an example session
where the work window contains gadgets with the Garnet look-and-feel.
The two types of gadget palette windows are shown in Figures
\ref{gadgetwindow} and \ref{motifgadgetwindow}.

For the Garnet look and feel, examples
are in Figures \ref{GadgetWindow}, \ref{CommandWindow} and
\ref{WorkWindow}.  Figure \ref{motifgadgetwindow} shows the Gadget
window for the Motif look and feel.

\begin{figure}
\bar{}
\begin{center}
\graphic{Postscript=`gilt/giltcommands.PS', boundingbox=file, magnify=.75}\end{center}
\caption{The Gilt Command window.  The `Edit' menu from the menubar provides
control over all properties of the gadgets in the work window and provides
dialog boxes for precise positioning.  Switching between `Build' and
`Run' mode allows you to test the gadgets as you build the interface.
Text boxes display the position and dimension of the selected gadget,
whose name appears at the bottom of the command window.}
\tag{CommandWindow}
\end{figure}

\begin{figure}
\bar{}
\begin{center}
\graphic{Postscript=`gilt/giltworkwin.PS', boundingbox=file, magnify=.75}\end{center}
\caption{The Gilt Work window showing a sample dialog box being
created using the Garnet look and feel.}
\tag{WorkWindow}
\bar{}
\end{figure}

\subsection{Gadget Palettes}

The single version of Gilt allows you to place Motif or Garnet look-and-feel
gadgets into your window (you can mix and match if you want).  To switch
back and forth, use the `\pr{Load Other Gadgets}' command in the `\pr{File}'
menu of the Gilt menubar.  You can only see one gadget window at a time.
The dialog boxes for Gilt itself use only the Motif look and feel.


\subsection{Placing Gadgets}

When you press with any mouse button on a gadget in the gadgets palette window
(see Figures \ref{GadgetWindow} or \ref{motifgadgetwindow}),
that gadget becomes selected.  Then, when you press with the {\it right}
mouse button in the work window, an instance of that gadget will be created.
Some gadgets, such as the scroll bars, have a variable size in one or more
dimensions, so for those you need to press the right button down, drag out
a region, and release the button.

The gadgets supplied for the {\it Garnet} look and feel are (from top to
bottom, left to right in Figure \ref{GadgetWindow}):
\index{Gadgets in Gilt}
\begin{itemize}
\item Menubar: a pull-down menu,

\item Text-button-panel: for commands,

\item Scrolling-menu: when there are many items to choose from,

\item Option-button: a popup-menu which changes the label of the button according to
the selected item,

\item Popup-menu-button: a popup-menu which does not change labels,

\item OK-Cancel: A special gadget to be used when you want the standard OK and
Cancel behavior (see section \ref{okcancel}),

\item X-button-panel: for settings where more than one is allowed,

\item Radio-button-panel: for settings where only one is allowed,

\item Menu: a menu with an optional title,

\item H-scroll-bar: for scrolling horizontally,

\item H-slider: for entering a number in a range,

\item V-scroll-bar: for scrolling vertically,

\item V-slider: for entering a number in a range,

\item OK-Apply-Cancel: Similar to OK-Cancel, but supports Apply (like OK,
but don't remove window),

\item Gauge: another way to enter a number in a range,

\item Trill-device: enter a number either in a range or not,

\item Labeled-box: enter any string; box grows if string is bigger,

\item Scrolling-labeled-box: enter a string; box has a fixed size and string
scrolls if too big,

\item Text: for decoration,

\item Multifont-text: for decoration,

\item Rectangle: for decoration,

\item Line: for decoration,

\item Bitmap: for decoration,

\item Pixmap: for decoration.
\end{itemize}


The gadgets supplied for the {\it Motif} look and feel are (from top to
bottom, left to right in Figure \ref{motifgadgetwindow}):

\begin{itemize}
\item Motif-Menubar: a pull-down menu

\item Motif-Text-button-panel: for commands,

\item Motif-Menu: a menu with an optional title,

\item Motif-Scrolling-Menu: when there are many items to select from,

\item Motif-Check-button-panel: for settings where more than one is allowed,

\item Motif-Radio-button-panel: for settings where only one is allowed,

\item Motif-OK-Cancel: A special gadget to be used when you want the standard OK and
Cancel behavior (see section \ref{okcancel}),

\item Motif-OK-Apply-Cancel: similar to OK-Cancel, but supports Apply,

\item Motif-Option-Buton: a popup-menu whose button's label changes according to the
selection

\item Motif-V-scroll-bar: for scrolling vertically,

\item Motif-V-slider: for entering a number in a range,

\item Motif-H-scroll-bar: for scrolling horizontally,

\item Motif-trill-device: for selecting from a range of numbers,

\item Motif-Gauge: another way to enter a number in a range,

\item Motif-Scrolling-labeled-box: enter a string; box has a fixed size and string
scrolls if too big,

\item Pixmap: for decoration

\item Bitmap: for decoration

\item Rectangle: for decoration,

\item Line: for decoration,

\item Motif-Box: A gadget that resembles a raised (or depressed) rectangle, used to
achieve a Motif style effect.  Set the \pr{:depressed-p} parameter.

\item Text: for decoration,

\item Multifont-text: for decoration,

\item Motif-Background: a special rectangle that helps achieve the Motif effect.
It always moves to the back of the window, and can only be selected at the
edges.
\end{itemize}

In addition to the standard gadgets, Gilt supplies a text string, a
line, a rectangle and a bitmap.  These are intended to be used as
decorations and
global labels in your dialog boxes.  They have no interactive behavior.

The Motif version also provides a background rectangle.   This is a
special rectangle which you should put behind your objects to make the
window be the correct color.  Note: to select the motif-background
rectangle, press at the edge of the window (the edge of the background
rectangle).  You might want to select the rectangle to delete it or
change its color (using the properties menu).


\subsection{Selecting and Editing Gadgets}

When you press with the left mouse button on a gadget in the work window,
it will become selected, and will show four or twelve selection handles.
The objects
that can change size (such as rectangles and scroll bars) display
black and white selection handles, and the objects that cannot change
size (such as buttons) only show white selection handles.\foot{You can
indirectly change the size of buttons by setting offsets and sizes in the
property sheet, however.}
If you press on a
{\it white} handle and drag, you can change the object's position.  If you
press on a {\it black} handle, you can change it's size (see Figure
\ref{handlesfig}).

\begin{figure}
\bar{}
\begin{center}
\graphic{Postscript=`gilt/gilthandles.PS', boundingbox=file}\end{center}
\caption{Pressing on a black selection handle causes the object to
grow, and pressing on a white one causes it to move.}
\tag{Handlesfig}
\bar{}
\end{figure}


If you press over an object with either the {\it middle} mouse button or
hold down the keyboard shift key while hitting the left button, then
that object is added to the selection set (so
you can get multiple items selected).  If you press with middle or
shift-left over an item that is {\it already} selected, then just that item
becomes de-selected.  If you press with the left button in the
background (over no objects), all objects are deselected.  While
multiple objects are selected, you can move them all as a group by
pressing on the selection handle of any of them.
Since Gilt uses the multi-selection gadget, it supports selecting
all objects in a region (hold down the left button and sweep out a
region), and growing multiple selected objects (if they are growable,
then press on a black handle at the outside of the set of objects).

To explicitly set the size or position of the selected object (when
only one is selected), you can use
the number fields in the Command Window (see Figure \ref{CommandWindow}).
Simply press with the left button in one of these fields and type a new
number.  When you hit \pr{return}, the object will be updated.  These fields
are a handy way to get objects to be evenly lined up (but also see the
`\pr{Align}' command).


\section{Editing Strings}

Editing the strings of most gadgets is straightforward: select the
gadget (to get the selection handles around it) and then click in a
string to get the string cursor, and then type the new string, and hit
\pr{return} when done.  If you make the string empty (e.g., by typing
\pr{control-u}),
and hit \pr{return}, that button of the gadget will be removed.  If you
edit the last item of the gadget and hit \pr{control-n} instead of \pr{return},
then a new item will be added to the gadget.  The strings can also be
edited by editing the \pr{:items} property in the property sheet that
appears from the \pr{Properties} command.

To edit string labels, simply click to select them, and then click
again with the left button to begin editing.  The fonts of
multifont strings can be edited using the keyboard commands described
in the `Multifont' section of the Opal Manual.

To edit the strings in a pop-up menu, like a menubar or an option
button, click once with the left button to select the gadget, and then
click again to pop-up the submenu.  You can now click in the submenu
to edit any of the items.  Use control-n in the last item to add new
items or control-u and return to remove items.  To edit the top-level
labels of a menubar, you need to click the left button three times:
once to select the gadget, once to bring up the submenu, and a third
time to begin editing.  Click outside to make the popped-up menu
disappear.

The editing operations supported for regular text (and labels) are:
\label{editingcommands}
\begin{description}
\item[] \pr{{\tt\char`\^}h, delete, backspace}: delete previous character.

\item[] \pr{{\tt\char`\^}w, {\tt\char`\^}backspace, {\tt\char`\^}delete}: delete previous word.

\item[] \pr{{\tt\char`\^}d}: delete next character.

\item[] \pr{{\tt\char`\^}u}: delete entire string.

\item[] \pr{{\tt\char`\^}b, left-arrow}: go back one character.

\item[] \pr{{\tt\char`\^}f, right-arrow}: go forward one character.

\item[] \pr{{\tt\char`\^}a}: go to beginning of the current line.

\item[] \pr{{\tt\char`\^}e}: go to end of the current line.

\item[] \pr{{\tt\char`\^}y}: insert the contents of the X cut buffer into the string at the
current point.

\item[] \pr{{\tt\char`\^}c}: copy the current string to the X cut buffer.

\item[] \pr{enter, return, {\tt\char`\^}j, {\tt\char`\^}J}: Finished.

\item[] \pr{{\tt\char`\^}n}: Finished, but add a new item (if a list).

\item[] \pr{line-feed}: Start a new line (if editing a multi-line text).

\item[] \pr{{\tt\char`\^}g}: Abort the edits and return the string to the way it was before
editing started.
\end{description}

\vspace{.5 line}
If the item is a member of a list, such as a menu item or a radio button,
then if the string is empty, that item will be removed.  If the string is
terminated by a \pr{{\tt\char`\^}n} (control-n) instead of by a return, and if this is the
last item, then a new item will be added.  The items can also be changed in
the properties dialog box for the gadget (see below).

Some strings cannot be edited directly, however.  This includes the
labels of sliders and gauges, and the indicators in scroll bars.  To
change these values, you have to use the property sheets.  Also,
for gadgets that have strings as their {\it values}, such as the text input
field and scrolling-text input field, you can only set the value
strings by going into Run mode.  Note, however, that the values are
not saved with the gadget (see section \ref{usinggiltdbs}).

To change the bitmap picture of a bitmap object, specify the name of
the new bitmap using the `\pr{Properties...}' command.


\section{Commands}

There are many commands in Gilt, and the command menu is a menubar
at the top of the main window.  The menubar implementation allows you to
give commands using keyboard shortcuts when the mouse is in the main
Work Window.
The particular shortcuts are listed on the sub-menus of the main
menubar.

The commands are:
\begin{description}
\item[] \pr{Cut} \dash remove the selected item(s) but save them in the clipboard
so they can later be pasted.

\item[] \pr{Copy} \dash copy the selected item(s) to the clipboard so they can
later be pasted.

\item[] \pr{Paste} \dash place a copy of the items in the clipboard onto the window.

\item[] \pr{Duplicate} \dash place a duplicate of the selected items onto the window.
(See section \ref{duplicating-objects}.)

\item[] \pr{Delete} \dash delete the selected objects and don't put them into
clipboard.  This operation can be undone with the \pr{Undo Last Delete}
command.  (See section \ref{deleting-objects}.)

\item[] \pr{Delete All} \dash delete all the objects in the window.  This
operation can be undone with the \pr{Undo Last Delete} command.
(See section \ref{deleting-objects}.)

\item[] \pr{Undo Last Delete} \dash undoes the last delete.  All the deletes are
saved, so this command can be executed multiple times to bring back
objects deleted earlier.  (See section \ref{deleting-objects}.)

\item[] \pr{Select All} \dash select all the objects in the window (including the
background object).

\item[] \pr{To Top} \dash make the selected objects not be covered by any other objects.
(See section \ref{to-top}.)

\item[] \pr{To Bottom} \dash make the selected objects be covered by all other objects.
(See section \ref{to-top}.)

\item[] \pr{Properties...} \dash bring up the properties window.  (See section
\ref{giltpropertiessec}).

\item[] \pr{Align} \dash bring up the dialog box to allow aligning of the selected
objects with respect to the first of the objects selected.
(See section \ref{align}.)
\end{description}

Many of these commands are now implemented with the functions in the
\pr{Standard-Edit} mechanism, described in the Gadgets Manual.


\subsection{To-Top and To-Bottom}
\label{to-top}

\index{To Top (in Gilt)}
\index{To Bottom (in Gilt)}
The selected object or objects can be made so they are not covered by
any objects using
the `\pr{To Top}' command in the Gilt Command Window.  The objects can be
made to be covered by all other objects by selecting the `\pr{To
Bottom}' command.


\subsection{Copying Objects}
\label{duplicating-objects}
\index{Duplicate (in Gilt)}

The `\pr{Duplicate}' command in the Command Window causes the selected
object or objects to
be duplicated.  The new object or objects will have all the same
properties as the
original, but the original and new objects can be subsequently edited
independently without affecting the other object
(the new object is a copy, not an {\it instance} of
the original).  The copy is placed at a fixed offset below and to the right
of the original, and is selected, so it can subsequently be moved.


\subsection{Aligning Objects}
\label{align}

\index{Align... (in Gilt)}

The Align function allows you to neatly line up a set of objects, and
to adjust their sizes to be the same.  Figure \ref{alignfig} shows the
dialog box that appears when the `\pr{Align...}' command is selected.  Align
adjusts the present positions of objects only; it does not set up
constraints.  Therefore, you can freely move objects after aligning them.

\begin{figure}
\bar{}
\begin{center}
\graphic{Postscript=`gilt/alignfig.PS', boundingbox=file, magnify=.75}\end{center}
\caption{The Align dialog box, after the user has specified that
the selected objects should be aligned and centered in a column and be
adjusted to be the same width.}
\tag{alignfig}
\bar{}
\end{figure}

To use Align, you first select two or more objects in the workspace
window (remember, to select more than one object, press on the objects
with the middle mouse button or hold down the shift key while hitting
the left button).  The {\it first} object you select is the reference
object, and the other objects will be adjusted with respect to that
first object.  For example, if you want to make objects be the same width,
then the width will be that of the first selected object.  You should
not change the selection while the Align dialog box is visible.

Aligning in a column or row also adjusts the spacing between objects
to be all the same.  The spacing used between the objects is the average
space between the objects before the command is given.

If a line is selected, then it is made to be exactly horizontal if
`Column' is specified, or vertical if `Row' is selected.  The size of
lines can also be adjusted using the width and height buttons.  If both
`Same Width' and `Same Height' are selected for a line, then an error
message is given.

If the `Same Width' and/or `Same Height' buttons are pressed, and one
of the selected objects other than the first cannot change size, then
an error message is presented.  All other selected objects are still
adjusted, however.


\begin{group}
\subsection{Deleting Objects}
\label{deleting-objects}

\index{Delete Selected (in Gilt)}
\index{Delete All (in Gilt)}
\index{Undo Last Delete (in Gilt)}
Choosing the `\pr{Delete Selected}' command in the Gilt Command Window will
remove the selected object or objects from the work window.  Selecting
the `\pr{Undo Last Delete}'
command will bring the object back.  Selecting `\pr{Delete All}' removes all
the objects from the work window.  `\pr{Undo Last Delete}' will bring all of
the objects back.  All of the deleted objects are kept in a queue, so the undo
command can be executed repeatedly.  Note that this is not a general Undo;
only undoing of deletes is supported.
\end{group}


\subsection{Properties}
\label{giltpropertiessec}
\index{Properties... (in Gilt)}

Each type of gadget has a number of properties.
First select the object in the workspace window, and then select the
`\pr{Properties...}' command (in the Gilt Command Window).
The window for the properties will appear below the selected object, but then
can be moved.  You should not change the selection while the
properties dialog box is visible.

If the selected object is a rectangle,
line or string, then special dialog boxes are available so you can
change the color, filling-style, font, etc.  These were created using Gilt.

The general property sheet lists all of the
properties that you can change, and will look something like Figure
\ref{propsheet}.  You can press in the value (right) side of any entry
and then type a new value (using the same editing commands as in
section \ref{editingcommands}).  You can move from field to field
using the tab key (after pressing with the left mouse button in a
field to start with).  When finished setting values, hit the `OK'
button to cause the values to be used and the property sheet to
disappear, or hit the `Apply' button to see the results and leave the
property sheet visible.  If you hit `Cancel', the changes will not
affect the object, and the property sheet will go away.

You can select multiple items and bring
up a property sheet on all of them.  The property sheet will show the
union of all properties of all objects.  If multiple objects have the
same property name, then the value of the property for the first
object selected is shown.

When you edit the value of a property and then hit return (or when you
hit OK for properties that pop up dialog boxes), the property sheet
will immediately set that property into all objects for which the
property is defined.  Thus, you can change the \pr{:foreground-color}
of all the objects by executing \pr{Select All}, bringing up the
\pr{Properties...}, and then editing the foreground-color property.
If you start to edit a property but change your mind, hit
\pr{Control-G} if text editing or \pr{Cancel} in a dialog box.  The
\pr{Done} button hides the property sheet.

The left, top, width and height number boxes displayed in the main
Gilt window will now also work on multiple objects.   When multiple
objects are selected, they show the values for the bounding box of all
the objects, and when you edit one and hit RETURN, that value is
applied to all objects for which it is settable.

For a complete explanation of what the fields of each
gadget do, see the Gadgets Manual.

\begin{figure}
\bar{}
\begin{center}
\graphic{Postscript=`gilt/giltpropsheet.PS', boundingbox=file, magnify=.75}\end{center}
\caption{The property sheet that appears for a particular X-Button-Panel.}
\tag{propsheet}
\bar{}
\end{figure}

Some of the fields of these property sheets are edited in a special way.  The
\pr{DIRECTION} field must be either \pr{:VERTICAL} or \pr{:HORIZONTAL}, so
the field shows these names, and you can press with the left button to pick
the desired value.  Fields that represent fonts show a special icon,
and if you click on it, the special font dialog box will appear.
However, the font is not changed in the object until the `OK' or
`Apply' buttons are hit on {\it both} the font dialog box and the main
property sheet.

\index{Known-as (in Gilt)}
The field named \pr{KNOWN-AS} should be set for
all gadgets that programs will want to know the values of, and will be the
name of the slot that holds the object (so it should be a keyword, e.g.,
\pr{:myvalue}).  The \pr{SELECT-FUNCTION} slot can contain a function to be
called at run time when the gadget is used.  Note that you might want to
specify the package name on the front of the function name.  However, if
you are going to have OK-Cancel or OK-Apply-Cancel in the dialog box, you
probably do not want to supply selection functions, since selection
functions are called when the gadget is used, not when OK is hit (see
section \ref{using}).

If the property sheet thinks any value is {\it illegal}, the value will
be displayed in italics after a return or tab is hit, and Gilt will
beep.  You can edit the value, or just leave it if the value will
become defined later (e.g., if the package is not yet defined).

Unfortunately, however, the error checking of the values typed into
the property sheets is not perfect, so be careful to check all the values
before hitting OK or Apply.  If a bad value is set into the gadget,
Gilt will crash.  You can usually recover from this by setting the
field back to a legal value in the Lisp window.  For example, if
\pr{:gray-width} got a bad value, you might type:
\begin{programexample}
(kr:s-value user::*gilt-obj* :gray-width 3)
(opal:update-all)
(inter:main-event-loop)
\end{programexample}
\index{*gilt-obj*}
\index{gilt-obj}

\subsection{Saving to a file}

\index{Save... (in Gilt)}
When the `\pr{Save...}' command is selected from the Command window, Gilt
pops up the dialog box shown in Figure \ref{Savedialogbox}.

\begin{figure}
\bar{}
\begin{center}
\graphic{Postscript=`gilt/savedialog.PS', boundingbox=file, magnify=.75}\end{center}
\caption{The dialog box that appears when the Save command is chosen.}
\tag{Savedialogbox}
\bar{}
\end{figure}

The only field you need to fill in is the `\pr{Filename}' field, which tells
the name of the file that should be written.  Simply press with the
left button in the field and begin typing.  This is a
scrollable field, so if the name gets too long, the text will scroll
left and right.  You might also want to use the window manager's cut
buffer ({\tt\char`\^}Y) if you
can select the string for the file in a different window.  Pressing
with the mouse button again will move the cursor, so you need to hit
\pr{return} or \pr{{\tt\char`\^}G} to stop editing the text field.

All the objects in the work window will be collected together in a
single Garnet ``aggregadget'' when written to the file.  The
`\pr{Top-level Gadget name}' field allows you to give this gadget a
name.  This is usually important if you want to use the gadget in some
interface, so you can have a name for it.  If you press the `\pr{Export
Top-level Gadget}' button, then an export line will be added to the
output file.

As described below in section \ref{Using}, there is a simple function for
displaying the created gadget in a window.  If you want this window to
have a special title, you can fill this into the `\pr{Window Title}' field.
The current position and size of the workspace window is
used to determine the default size and position of the dialog box
window when it is popped up, so you should change the workspace
window's size and position (using the standard window manager
mechanisms) before hitting OK in the Save dialog box.

If you want the gadget to be defined in a Lisp package other than \pr{USER},
then you can fill this into the `\pr{Package name}' field.

Finally, if you have included the special OK-Cancel gadget in your
workspace window, then the `\pr{Function-for-OK name}' field will be
available.  Type here the name of the function you want to have called
when the OK button is hit.  The parameters to this function are
described in section \ref{Using}.

After filling in all the fields, hit `OK' to actually save the file, or
`Cancel' to abort and not do the save.

If you have already read or saved a file, then the values in the Save
dialog box will be based on the previous values.  Otherwise, the
system defaults will be shown.

{\bf Note: There is no protection or confirmation required before
overwriting an existing file.}


\subsection{Reading from a file}

\index{Read... (in Gilt)}
You can read files back into Gilt using the `\pr{Read...}' command.
This displays the dialog box shown in Figure \ref{readdialogbox}.
Press with the left mouse button in the `\pr{Filename}' field and type
the name of the file to be read, then hit return.

\begin{figure}
\bar{}
\begin{center}
\graphic{Postscript=`gilt/readdialog.PS', boundingbox=file, magnify=.75}\end{center}
\caption{The dialog box that appears when the Read command is hit.}
\tag{readdialogbox}
\bar{}
\end{figure}

If there are objects already in the workspace window, then you have
the option of adding the objects in the file to the ones already in
the work window using the `\pr{Add to existing objects}' option, or else
you can have the contents of the workspace window deleted first using
the `\pr{Replace existing objects}' option.  If you use the `\pr{Replace}'
option, then the window size is adjusted to the size specified when
the file was written.  Also, reading a file using the replace option puts the
previous contents of the workspace window in the delete stack so
that they can be retrieved using the `\pr{Undo Last Delete}' command.

Output produced from the GarnetDraw utility program can be read into Gilt,
which would allow more elaborate decorations to be added to a dialog box.
But in general, only files written with Gilt can be read with Gilt.


\subsection{Value and Enable Control}

A sophisticated module for modifying the values of gadgets in the Gilt
work-window has been added, along with a corresponding module to modify
when a gadget should be active (or grayed-out).  These are called
the \pr{Value Control} and \pr{Enable Control} modules, and can be invoked
from the `Control' submenu in the Gilt menubar.

These modules implement the ideas discussed in \cite{GiltDemo}.
The paper includes examples of how to use this feature, but
a full set of documentation is still pending.  If there is sufficient demand
for documentation of this module, we will supply an addendum to this manual
(direct requests to \pr{garnetjcs.cmu.edu}).


\section{Run Mode}

\index{Run (in Gilt)}
\index{Build (in Gilt)}

To try out the interface, just click on the button in the command
window labeled `Run'.  This will grey out most of the commands, and
allow the gadgets in the work window to execute as they will for the
end user (except that application functions will not be called).  To
leave run mode, simply press on the `Build' button.


\section{Hacking Objects}

\index{*gilt-obj*}
\index{gilt-obj}
Gilt does not provide all options for all objects and gadgets.  If you
want to change other properties of objects that are not available from
the property sheets, you could hit the HELP key while the mouse is positioned
over the object to bring up the Inspector (see the Debugging manual, starting
on page \value{debug} for details).

You can also access the selected object directly from
Lisp.  If one object is selected, its name is printed in the command
window.  Also the variable \pr{user::*gilt-obj*} is set with the single
selected object.  If multiple objects are selected, then
\pr{user::*gilt-obj*} is set with the list of objects selected.
You can go into the Lisp listener window, and type
Garnet commands to affect the selected object (e.g., \pr{s-value} some
slots), and call \pr{(opal:update-all)}.  This
technique can also be used to add extra slots to objects.  The
changes you make will be saved with the object when it is written.



\section{Using Gilt-Created Dialog Boxes}
\label{usinggiltdbs}

There are various ways to use Gilt-created collections of gadgets in
an application.

The file that Gilt creates is a normal Lisp text file that creates the
appropriate Garnet objects when loaded.  The file should be compiled
along with your other application files, in order to provide better
performance.


\subsection{Pop-up dialog box}
\label{popupdialogbox}
\label{okcancel}

\index{OK-Cancel gadget (in Gilt)}
\index{OK-Apply-Cancel gadget (in Gilt)}
\index{Pop-Up Dialog Boxes (from Gilt)}

Probably the easiest way to use a set of gadgets is as a pop-up dialog
box.  The application should be sure to {\it load} the file that Gilt
created before calling the functions below.

When Gilt writes out the gadgets, it does {\it not} save the values as
the initial defaults.  Therefore, if you want to have default values
for any gadgets, you need to set them from your program.  This should
be done {\it before} the window is
displayed using the function:
\index{Set-Initial-Value (in Gilt)}
\begin{programexample}
gilt:Set-Initial-Value {\it top-gadget gadget-name value}\value{function}
\end{programexample}
The \pr{top-gadget} is the top-level gadget name specified in the
`\pr{Top-level Gadget name}' field of the Save dialog box.  The
\pr{gadget-name} is the name of the particular gadget to be
initialized.  This name will be a keyword, and will have been
specified as the \pr{KNOWN-AS} property of the gadget using the
gadget's property sheet (which appears when you hit the
`\pr{Properties...}' command).  The \pr{value} is the value to be used as
the default, in the appropriate format for that gadget.

Next, the Gilt function \pr{show-in-window} can be used to display the
dialog box in a window:
\index{Show-In-Window (in Gilt)}
\begin{programexample}
gilt:Show-In-Window {\it top-gadget} \&optional {\it x y modal-p}\value{function}
\end{programexample}
The {\it top-gadget} is the gadget name used in the Save dialog box.
The size of the window is determined by the size of the workspace
window when the file was written.  The position of the window will
either be the position when written, or it can be specified as the
{\it x} and {\it y} parameters, which are relative to the screen's
upper-left corner.  When the {\it modal-p} parameter is T, then interaction in
all other Garnet windows will be suspended until the window goes away (e.g.,
when the user clicks the `OK' button).  If you want the window relative to a
position in another window, the function \pr{opal:convert-coordinates} is
useful.

The function \pr{show-in-window-and-wait} performs the same function
as \pr{show-in-window}, but it waits for the user to click on an OK or
Cancel button before returning (\pr{show-in-window} returns
immediately after bringing up the window).
\index{Show-In-Window-And-Wait (in Gilt)}
\begin{programexample}
gilt:Show-In-Window-And-Wait {\it top-gadget} \&optional {\it x y modal-p}\value{function}
\end{programexample}
When the user clicks on the OK button, this function will return the
values of all the gadgets in the dialog box in the form of
\pr{gilt:gadgets-values}, which is:
\begin{programexample}
((:FILENAME `/usr/bam/garnet/t1.lisp') (:VAL 49) (:BUTTON `Start'))
\end{programexample}
where the keywords are the names (\pr{:known-as} slot) of the gadgets.
If the user hits Cancel, then \pr{show-in-window-and-wait}
returns NIL.  Apply does not cause the dialog box to go away, so you
might want to supply an OK-Function for the dialog box.

If selection functions were specified in the gadget's select-function
slot using the `\pr{Properties...}' command, then these functions are
called immediately when the gadgets are used.

If the dialog box has an OK-Cancel or OK-Apply-Cancel gadget in it,
then the function
specified in the `\pr{Function-For-OK name}' field of the Save dialog
box will be called when the user hits the OK or Apply buttons.  This function
is parameterized as:
\index{Function-For-OK (in Gilt)}
\begin{programexample}
(lambda (top-gadget values)
\end{programexample}
The \pr{top-gadget} is the same as above.  The \pr{values} parameter
will be a list containing pairs of all the gadget names of gadgets
which have names, and
the value of that gadget.  Again, the names are the keywords supplied
to the \pr{KNOWN-AS} property.  For example, \pr{values} might
contain:
\begin{programexample}
((:FILENAME `/usr/bam/garnet/test1') (:REINITIALIZE NIL))
\end{programexample}

The function
\index{Value-Of (in Gilt)}
\begin{programexample}
gilt:Value-Of {\it gadget-name values}\value{function}
\end{programexample}
can be used to return the value of the specific gadget named
\pr{gadget-name} from the values list \pr{values}.  For example, if
\pr{v} is the above list, then \pr{(gilt:value-of :filename v)} would
return \pr{`/usr/bam/garnet/test1'}.

After the Function-For-OK is called, the dialog box window is made
invisible if OK was hit, and left in place if Apply was hit.  If
Cancel was hit, then the window is simply
made invisible.  If \pr{show-in-window} is called again on the same
dialog box, the old window is reused, saving time and space.

To destroy the window associated with a gadget, use the function:
\index{Destroy-Gadget-Window (in Gilt)}
\begin{programexample}
gilt:Destroy-Gadget-Window {\it top-gadget}\value{function}
\end{programexample}
This does {\it not} destroy the \pr{top-gadget} itself.  Note that
destroying the top-gadget while the window is displayed will not
destroy the window.  However, destroying the window explicitly (using
\pr{opal:destroy}) {\it will} destroy both the window and the gadget.

\section{Using Gilt-Created Objects in Windows}
\label{using}

If you want to use the gilt-created gadgets inside of an application
window, you only need to create an instance of the top-gadget, which
is the top-level gadget name specified in the
`\pr{Top-level Gadget name}' field of the Save dialog box.  The
instance will have the same position in the application window as it had
in the Gilt workspace window.  If you use the
standard Gilt OK-Cancel gadget, it will make the application
window be invisible when the OK or Cancel buttons are hit.  If you do
not want this behavior, then you need to create your own OK-Cancel
buttons.

The \pr{set-initial-value} described above can still be used for gilt
gadgets in application windows.  In addition, the function
\index{Gadget-Values (in Gilt)}
\begin{programexample}
gilt:Gadget-Values {\it top-gadget}\value{function}
\end{programexample}
can be used to return the values list of all gadgets with names.  The
return value is in the form of the \pr{values} parameter passed to the
Ok-Function.



\section{Hacking Gilt-Created Files}

Since the file that Gilt creates is a normal text file, it is possible
to edit it with a normal text editor.  Some care must be taken when
doing this, however, if you want Gilt to be able to read the file back
in for further editing.  (If you do not care about reading the files
back in, then you can edit the file however you like.)

The simplest changes are to edit the values of slots of the objects.
These edits will be preserved when the file is read in and written
back out.  Be sure not to change the value of the \pr{:gilt-ref} slot.

When Gilt saves objects to a file, it sets the \pr{:constant} slot of
all the gadgets to T.  If you expect to ever change any properties of
widgets in the dialog boxes when they are being used by an
application, then you should hand-edit the Gilt-generated file to
change the constant value (typically by \pr{:except}ing the slots you
plan to change dynamically).  (Gilt reads in the file using
\pr{with-constants-disabled}, so the defined constant slots will not
bother Gilt.)

If you want to create new objects, then these can be put into the top
level aggregadget definition.
You should follow the convention of having a \pr{:box} (or
\pr{:points}) slot and
putting the standard constraints into the \pr{:left} and \pr{:top}
fields (or \pr{:x1, :y1, :x2} and \pr{:y2}).  For example, to add a
circle, the following code might be
added into the top-level aggregadget's \pr{:parts} list:
\begin{programexample}
    (:mycircle ,opal:circle
      (:box (50 67 30 30))
      (:left ,(o-formula (first (kr:gvl :box))))
      (:top ,(o-formula (second (kr:gvl :box))))
      (:width 30)
      (:height 30))
\end{programexample}
\index{gilt-ref slot (in Gilt)}
If you do not supply a \pr{:gilt-ref} field, Gilt will allow
the user to move the object around, but not
change its size or any other properties.  For some objects, it might
work to specify the \pr{:gilt-ref} slot as \pr{`TYPE-RECTANGLE'},
\pr{`TYPE-LINE'} or \pr{`TYPE-TEXT'}.

If you add extra functions or comments to the file, they will {\it not}
be preserved if the file is written out and read in.  Similarly,
interactors added to the top level gadget will not be preserved.

\chapter*{References}
\bibliography
